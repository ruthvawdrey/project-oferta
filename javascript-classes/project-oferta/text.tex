$

make the amazon project - today

___________________________________________________________________________________________________


Logic Puzzle:

You are standing in a room with three light switches. 
The switches all correspond to three different light bulbs in an adjacent room that you cannot see into.
 With all the light switches starting in the off position,
 how can you find out which switch connects to which light bulb?

 Answer:
You may only enter the room with the light bulbs one time.


1. Turn on the first switch and leave it on for a few minutes.
2. After a few minutes, turn off the first switch and turn on the second switch.
3. Immediately go into the room with the light bulbs.
4. The bulb that is on corresponds to the second switch (the one you left on).
5. The bulb that is off but warm corresponds to the first switch (the one you turned on first and then off).
6. The bulb that is off and cold corresponds to the third switch (the one you never turned on).    

___________________________________________________________________________________________________


Browser Storage

Local Storage and Session Storage are two types of web storage that allow websites to store data on a user's browser. They are part of the Web Storage API and provide a way to store key-value pairs in a web browser.

Local Storage:
- Data persists even when the browser is closed and reopened.
- Suitable for storing non-sensitive data that needs to be available across sessions.

Session Storage:
- Data is only available for the duration of the page session.
- Suitable for storing temporary data that should not persist after the tab is closed.
- Data is cleared when the page session ends (i.e., when the tab or browser is closed).

___________________________________________________________________________________________________


Cookies in browser storage
Cookies are small pieces of data that are stored on a user's computer by their web browser while they are browsing a website. They are used to remember information about the user, such as login credentials, preferences, and tracking information.   

small pieces of data
stored on a user's computer
remember information about the user
login credentials
preferences
tracking information    
(steal information)
Request on the server side
Cookies are sent to the server with every HTTP request, allowing the server to identify the user and retrieve their stored information.
stored user name
Track analytics

___________________________________________________________________________________________________
console 
document.cookie
// Set a cookie
document.cookie = "username=JohnDoe; expires=Fri, 31 Dec 2024 23:59:59 GMT; path=/";

___________________________________________________________________________________________________

Local Storage
// Set an item in local storage
localStorage.setItem("username", "JohnDoe");    

Safe data

// Store data in localStorage
localStorage.setItem("key", "value");

// Retrieve data from localStorage
const value = localStorage.getItem("key");

// Remove data from localStorage
localStorage.removeItem("key");

// Clear all data from localStorage
localStorage.clear();


localStorage.setItem("username", JSON.stringify({ name: "John", age: 30 }));

const retrievedData = JSON.parse(localStorage.getItem("username"));

console.log(retrievedData.name); // Output: John
console.log(retrievedData.age);  // Output: 30

localStorage.clear() // clear all data
localStorage.removeItem("username") // remove specific item



setItem
getItem
removeItem -----
clear   

JSON.parse(localStorage.getItem("username") )

JSON.stringify      


const updateItems = (key, newValue) => {
  const existingData = JSON.parse(localStorage.getItem(key)) || {};
  const updatedData = { ...existingData, ...newValue };
  localStorage.setItem(key, JSON.stringify(updatedData));


  const updateItems = (cartItems) => {
    const updatedItems = cartItems.filter((item) => item.name !== "Apple");
    % localStorage.setItem("cart", JSON.stringify(updatedItems));
  };


  function for delete item inside of a cart 


    const deleteItem = (itemName) => {
        const cartItems = JSON.parse(localStorage.getItem("cart")) || [];
        const updatedItems = cartItems.filter((item) => item.name !== itemName);
        localStorage.setItem("cart", JSON.stringify(updatedItems));
    };

___________________________________________________________________________________________________


  console 

  localStorage 

  ___________________________________________________________________________________________________

don't safe safe information
password
credit card
social security number 
delete cookies 

delete all your safe password from browser
___________________________________________________________________________________________________

session Storage
// Set an item in session storage
sessionStorage.setItem("sessionID", "abc123");  

session-based storage

// Store data in sessionStorage
sessionStorage.setItem("key", "value");



shopping cart
// Retrieve data from sessionStorage
const value = sessionStorage.getItem("key");

___________________________________________________________________________________________________


IndexedDB
IndexedDB is a low-level API for client-side storage of significant amounts of structured data, including files/blobs. 
This API uses indexes to enable high-performance searches of this data. While Web Storage


They don't need internet to access data

NoSQL database
store significant amounts of data
structured data
files/blobs
high-performance searches
key-value pairs
asynchronous API
transactional database systems (localStorage and sessionStorage) are useful for storing small amounts of data, IndexedDB is more powerful and suitable for applications that require storing large amounts of data.
___________________________________________________________________________________________________

Understand Backend for security



security concerns : XSS attacks
Cross-Site Scripting (XSS) attacks are a type of security vulnerability 
that allows attackers to inject malicious scripts into web pages viewed by other users.
 These attacks can lead to various harmful consequences, including data theft, session hijacking, and defacement of websites. 
 Here are some key points
to understand about XSS attacks:
1. Types of XSS Attacks:
   - Stored XSS: The malicious script is permanently stored on the target server, such as in a database, and is served to users when they access the affected page.
   - Reflected XSS: The malicious script is reflected off a web server, typically in an error message or search result, and is immediately executed in the user's browser.
   - DOM-based XSS: The vulnerability exists in the client-side code rather than the server-side code. The malicious script manipulates the DOM (Document Object Model) of the page.
2. Consequences of XSS Attacks:
   - Data Theft: Attackers can steal sensitive information such as cookies, session tokens, or other personal data.
   - Session Hijacking: Attackers can take over a user's session, gaining unauthorized access to their account.
   - Defacement: Attackers can modify the content of a website, often to display unwanted or harmful content.
   - Malware Distribution: Attackers can use X  SS to distribute malware to users.
3. Prevention Measures:
   - Input Validation: Always validate and sanitize user inputs to ensure they do not contain malicious code.
   - Output Encoding: Encode data before rendering it in the browser to prevent the execution of malicious scripts.
   - Content Security Policy (CSP): Implement CSP headers to restrict the sources from which    scripts can be loaded and executed.
   - Use Security Libraries: Utilize libraries and frameworks that have built-in protections against XSS attacks.
   - Regular Security Audits: Conduct regular security assessments and code reviews to identify and fix vulnerabilities.
4. User Awareness:
   - Educate users about the risks of XSS attacks and encourage them to report suspicious activities.
   - Advise users to keep their browsers and plugins up to date to mitigate vulnerabilities.        
5. Browser Security Features:
   - Modern browsers have built-in security features that can help mitigate the risks of XSS attacks, such as same-origin policies and XSS filters.
Overall, preventing XSS attacks requires a combination of secure coding practices, user education, and the use of security tools and features.



Mitigating security risks in web applications involves implementing a combination of best practices, tools, and strategies to protect against various threats. Here are some key measures to consider:  
1. Input Validation and Sanitization:
   - Always validate and sanitize user inputs to prevent injection attacks, such as SQL injection and XSS (Cross-Site Scripting).
   - Use libraries or frameworks that provide built-in input validation features.
2. Use HTTPS:
   - Ensure that your web application uses HTTPS to encrypt data transmitted between the client and server.
   - Obtain and regularly renew SSL/TLS certificates.
3. Implement Strong Authentication and Authorization:
   - Use strong password policies and multi-factor authentication (MFA) to enhance security.                



   avoid storing sensitive information in browser storage
   use encryption for sensitive data

   implement input validation and sanitization
   use output encoding to prevent XSS attacks
   use HttpOnly and Secure flags for cookies
   implement Content Security Policy (CSP)
   regularly update and patch software
   conduct regular security audits and penetration testing
   educate users about security best practices
   monitor and log suspicious activities
   use security headers (e.g., X-Content-Type-Options, X-Frame-Options)
   limit data exposure and access controls
   back up data regularly and securely
   use web application firewalls (WAF)
   stay informed about the latest security threats and vulnerabilities
   follow the principle of least privilege for user roles and permissions
   implement session management best practices (e.g., session expiration, secure cookies)
   ensure proper error handling to avoid information leakage        

    use security-focused libraries and frameworks
    keep dependencies up to date    

    implement rate limiting to prevent brute-force attacks

4. Secure Session Management:
   - Use secure cookies with the HttpOnly and Secure flags to protect session data.
   - Implement session expiration and regeneration mechanisms.
5. Regularly Update and Patch Software:
   - Keep your web server, database, and application frameworks up to date with the latest security patches.
   - Monitor for vulnerabilities in third-party libraries and dependencies.
6. Implement Content Security Policy (CSP):
   - Use CSP headers to restrict the sources from which content can be loaded, reducing the risk of XSS attacks.
7. Use Security Headers:        
   - Implement security headers such as X-Content-Type-Options, X-Frame-Options, and X-XSS-Protection to enhance security.
8. Conduct Regular Security Audits and Penetration Testing:
   - Regularly assess your web application for vulnerabilities through security audits and penetration testing.
   - Address any identified issues promptly.
9. Educate Users About Security Best Practices:
   - Provide training and resources to help users recognize phishing attempts and other social engineering attacks.
   - Encourage strong password practices and the use of password managers.
10. Monitor and Log Suspicious Activities:
    - Implement logging and monitoring to detect and respond to suspicious activities in real-time.
    - Use tools like intrusion detection systems (IDS) and security information and event management (SIEM) solutions.
11. Limit Data Exposure and Access Controls:
    - Follow the principle of least privilege by granting users and services only the access they need.
    - Regularly review and update access controls.
12. Back Up Data Regularly and Securely:
    - Implement regular data backups and ensure they are stored securely.
    - Test backup and recovery processes periodically.
13. Use Web Application Firewalls (WAF):
    - Deploy a WAF to filter and monitor HTTP traffic to and from your web application.
    - Configure the WAF to block common attack patterns.        
14. Stay Informed About the Latest Security Threats and Vulnerabilities:
    - Subscribe to security bulletins and follow industry news to stay updated on emerging threats.
    - Participate in security communities and forums.
15. Follow the Principle of Least Privilege for User Roles and Permissions:
    - Ensure that users and services have only the minimum permissions necessary to perform their tasks.
    - Regularly review and adjust permissions as needed.        


    Perform considerations for data sensitivity and compliance requirements
    Implement rate limiting to prevent brute-force attacks
    Use security-focused libraries and frameworks
    Keep dependencies up to date
___________________________________________________________________________________________________     

Storage Limitations 

Pitforms	Storage Limitations
Local Storage	Typically around 5-10 MB per origin (varies by browser)
Session Storage	Typically around 5-10 MB per origin (varies by browser)

___________________________________________________________________________________________________     

codehs.com