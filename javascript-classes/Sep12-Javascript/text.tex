To install an autocomplete AI tool for LaTeX in VS Code, follow these steps:

1. Open VS Code.
2. Go to Extensions (Ctrl+Shift+X).
3. Search for "LaTeX Workshop" and install it.
4. For AI-powered autocomplete, install "GitHub Copilot":
    - Search for "GitHub Copilot" in Extensions.
    - Click Install.
5. Sign in to GitHub when prompted to activate Copilot.

Now, Copilot will provide AI-powered autocomplete suggestions while you write LaTeX.

% To change text color in LaTeX, use the xcolor package.
\usepackage{xcolor}

% Example: Change text color to red
\textcolor{red}{This text is red.}

% You can also define custom colors:
\definecolor{myblue}{RGB}{30, 144, 255}
\textcolor{myblue}{This text uses a custom blue color.}


$
Javascript Numbers and Math
_______________________________
method = function 
 __________________________________

callback - function
  __________________________________

declare 

 __________________________________

return = end function
 __________________________________


 var = function-scoped 
 __________________________________

In JavaScript, you can declare a constant using the `const` keyword. For example:

\begin{verbatim}
const pi = 3.14159;
\end{verbatim}

This creates a constant named \texttt{pi} whose value cannot be changed.


_________________________________________________________

Javascript Numbers and Math


In JavaScript, an \textbf{integer} is a number without a decimal point. For example:

\begin{verbatim}
let count = 10; // integer
let price = 4.99; // not an integer (has decimals)
\end{verbatim}

JavaScript does not have a separate integer type; all numbers are represented as \texttt{Number} type, which can be integers or floating-point values.


A \textbf{float} (short for "floating-point number") in JavaScript is a number that has a decimal point. For example:

\begin{verbatim}
let temperature = 23.5; // float
let pi = 3.14159; // float
\end{verbatim}

In JavaScript, both integers and floats are represented by the same \texttt{Number} type. There is no separate type for floating-point numbers.

+ operators
mode

modulos
give us the remainder

Exponentiation

___________________________________

let num = 4;
console.log(num);

console.log(++num);

// console.log(num++) use the variable and increment after

button.addEventListener("click", function (){
    this.innrHTML="Clicked" + ++count +"times";
    console.log(count);
});
__________________________________________

// IEEE 754 Standart
Floating-point numbers in JavaScript follow the IEEE 754 standard, which can lead to precision issues. For example:

\begin{verbatim}
console.log(0.1 + 0.2); // Output: 0.30000000000000004
\end{verbatim}

This happens because some decimal fractions cannot be represented exactly in binary floating-point. Always be careful when comparing floating-point numbers for equality.
__________________

Math in the console

Math.pi

Math.round(x)
Math.random()

generate unique ID

The \texttt{Math.random()} function in JavaScript returns a pseudo-random floating-point number in the range $[0, 1)$ (inclusive of $0$, but not $1$). It is commonly used for generating random numbers, such as for games, simulations, or creating unique identifiers.

\begin{verbatim}
let randomNumber = Math.random();
console.log(randomNumber); // Example output: 0.726357843
\end{verbatim}

To generate a random integer within a specific range, you can use:

\begin{verbatim}
// Random integer between min (inclusive) and max (exclusive)
function getRandomInt(min, max) {
    return Math.floor(Math.random() * (max - min)) + min;
}
\end{verbatim}

\textbf{Note:} \texttt{Math.random()} is not suitable for cryptographic purposes.


Math.ceil(x) ========= write up
Math.floor(x) ========= write down

Math.truc(x)



Math.abs(x) =========== absolute value

Math.sign(x)============== sign of the number

Math.pow(x,y)============ power

Math.sqrt(x))============ sqrt

Math.min() ===== arguments minimum 
Math.max()

Math.random()


password generation

Math.floor(Math.random()*10)
________________________________________

5 == 5 

expression return a value


statement 
const num = 6 + 2

________________________________________

5 === '5' the type 
5 == '5' value
 5 != 5 not ( boolean)
5 !=='5' trueee


________________________________________

amazon order sumary
calculate amazon car
calculate tax

________________________________________

